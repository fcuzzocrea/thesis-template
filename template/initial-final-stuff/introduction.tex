%L’introduzione deve essere atomica, quindi non deve contenere né sottosezioni né paragrafi né altro. 
%Il titolo, il sommario e l’introduzione devono sembrare e e delle scatole cinesi, nel senso che lette in quest’ordine devono progressivamente svelare informazioni sul contenuto per incatenare l’attenzione del lettore e indurlo a leggere l’opera fino in fondo. 
%L’introduzione deve essere tripartita, non graficamente ma logicamente.

%Inquadramento generale
%La prima parte contiene una frase che spiega l’area generale dove si svolge il lavoro; una che spiega la sottoarea più specifica dove si svolge il lavoro e la terza, che dovrebbe cominciare con le seguenti parole «lo scopo della tesi è …», illustra l’obbiettivo del lavoro. 
%Poi vi devono essere una o due e frasi che contengano una breve spiegazione di cosa e come è stato fatto, e delle attività sperimentali, dei risultati ottenuti con una valutazione e degli a sviluppi futuri. 
%La prima parte deve essere circa una facciata e mezza o due.

%Breve descrizione del lavoro
%La seconda parte deve essere una esplosione della prima e deve quindi mostrare in maniera pi` esplicita l’area dove si svolge il lavoro, le fonti u bibliografiche pi` importanti su cui si fonda il lavoro in maniera sintetica u (una pagina) evidenziando i lavori in letteratura che presentano attinenza con il lavoro affrontato in modo da mostrare da dove e perché è sorta la tematica di studio. 
%Poi si mostrano esplicitamente le realizzazioni, le direttive future di ricerca, quali sono i problemi aperti e quali quelli affrontati e si ripete lo scopo della tesi. 
%Questa parte deve essere piena di citazioni bibliografiche e deve essere lunga circa 4 facciate.

%Struttura della tesi
%La terza parte contiene la descrizione della struttura della tesi ed è organizzata nel modo seguente. 
%«La tesi è strutturata nel modo seguente». Nella sezione due si mostra…, nella sezione tre si illustra… .