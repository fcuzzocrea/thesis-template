%The conclusions must recall the field of work, the purpose of the thesis, what has been done and an evaluation of the obtained results.
%Furthermore, the conclusion must also emphasize the future prospects and must show how to move forward in the study area.
\begin{quotation}
{\footnotesize
\noindent{\emph{``If you didn't get angry and mad and frustrated, that means you don't care about the end result, and are doing something wrong.''\\}
}
\begin{flushright}
Greg Kroah-Hartman
\end{flushright}
}
\end{quotation}
\vspace{0.5cm}

 Some of the \acrshort{povray} major drawbacks when applied to spaceborne imagery generation are \cite{pangufinal}:

\begin{itemize}
  \item only a Lambertian reflectance model is possible;
  \item a uniform surface albedo is used and realistic albedo values cannot used;
  \item the extended illumination source (sun) can be modeled only as an array of point light sources or as an area light, which is a suboptimal solution;
  \item background lighting (starlight) is cannot modeled;
  \item earthshine cannot be easily modeled;
  \item \acrshort{povray} can produce high quality images but rendering is slow as many minutes (sometimes hours) are required to produce a single image, depending on the H/W is running on.
\end{itemize}

Finalmente è finita.
