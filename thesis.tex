%% LyX 2.3.5.2 created this file.  For more info, see http://www.lyx.org/.
%% Do not edit unless you really know what you are doing.
\documentclass[11pt,english,openright]{book}
\usepackage[T1]{fontenc}
\usepackage[latin9]{inputenc}
\usepackage[a4paper]{geometry}
\geometry{verbose,tmargin=3cm,bmargin=3.5cm,lmargin=4cm,rmargin=3cm}
\setcounter{secnumdepth}{3}
\setcounter{tocdepth}{3}
\usepackage{color}
\usepackage{babel}
\usepackage[useregional]{datetime2}

\usepackage{float}
\usepackage{booktabs}
\usepackage{url}
\usepackage{amsmath}
\usepackage{amssymb}
\usepackage{graphicx}
\usepackage{setspace}
\onehalfspacing
\usepackage[unicode=true,pdfusetitle,
 bookmarks=true,bookmarksnumbered=false,bookmarksopen=false,
 breaklinks=false,pdfborder={0 0 1},backref=false,colorlinks=false]
 {hyperref}

\makeatletter

%%%%%%%%%%%%%%%%%%%%%%%%%%%%%% LyX specific LaTeX commands.
\providecommand{\LyX}{\texorpdfstring%
  {L\kern-.1667em\lower.25em\hbox{Y}\kern-.125emX\@}
  {LyX}}
\DeclareRobustCommand*{\lyxarrow}{%
\@ifstar
{\leavevmode\,$\triangleleft$\,\allowbreak}
{\leavevmode\,$\triangleright$\,\allowbreak}}
%% Because html converters don't know tabularnewline
\providecommand{\tabularnewline}{\\}
\floatstyle{ruled}
\newfloat{algorithm}{tbp}{loa}[chapter]
\providecommand{\algorithmname}{Algorithm}
\floatname{algorithm}{\protect\algorithmname}

%%%%%%%%%%%%%%%%%%%%%%%%%%%%%% User specified LaTeX commands.
% additional packages
\usepackage{tabularx}
\usepackage{setspace}
\usepackage{amsthm}
\usepackage{rotating}
\usepackage{caption}
\usepackage{epsfig}
\usepackage{indentfirst}
\usepackage{fancyhdr}
\usepackage{url}
\usepackage{cite}
\usepackage[normalem]{ulem}
\usepackage[table]{xcolor}
\usepackage{booktabs}
\usepackage{algpseudocode}
\usepackage[detect-all]{siunitx}
\usepackage{euler}
\usepackage{sectsty}
\usepackage[font={footnotesize }]{caption}
\usepackage{multicol}
\usepackage{prettyref}
\usepackage[nointegrals]{wasysym}

% Elenco degli acronimi
\usepackage[acronym,nonumberlist,nopostdot]{glossaries}
\makenoidxglossaries

% Definire qui gli acronimi che verranno utilizzati nel testo
\newacronym{sc}{S/C}{Spacecraft}
\newacronym{oos}{OOS}{On-Orbit Servicing}
\newacronym{adr}{ADR}{Active Debris Removal}
\newacronym{ff}{FF}{Formation Fliying}
\newacronym{lidar}{LIDaR}{Light Detection and Ranging}
\newacronym{floss}{FLOSS}{Free, Libre and Open Source Software}
\newacronym{cv}{CV}{Computer-Vision}
\newacronym{svd}{SVD}{Sharma-Ventura-D'Amico}
\newacronym{pnp}{P-\textit{n}-P}{Perspective-\textit{n}-Point}
\newacronym{2d}{2-D}{Two-Dimensional}
\newacronym{3d}{3-D}{Three-Dimensional}
\newacronym{dcm}{DCM}{Direction Cosine Matrix}
\newacronym{slam}{SLAM}{Simultaneous Localization and Mapping}
\newacronym{cg}{CG}{Center of Mass}
\newacronym{rd}{R\&D}{Research and Development}
\newacronym{wge}{WGE}{Weak Gradient Eliminator}
\newacronym{nr}{NR}{Netwon-Raphson}
\newacronym{eo}{EO}{Electro-Optical}
\newacronym{stl}{STL}{Stereolitographic}

% Definire qui i simboli che verranno utilizzati nel testo
\newglossaryentry{t_c}{name=${t_C}$,description={Target Center of Mass Location WRT Camera Frame}}
\newglossaryentry{A_TC}{name=${A_{BC}}$,description={Orientation of Target Principal Axes WRT Camera Frame}}
\newglossaryentry{fx}{name=${f_x}$,description={Orizontal Focal Length}}
\newglossaryentry{fy}{name=${f_y}$,description={Vertical Focal Length}}

% fixes the page number of the first page of each chapter
\fancypagestyle{plain}{
\fancyhead{}
\renewcommand{\headrulewidth}{0pt}
\renewcommand{\footrulewidth}{0pt}
\fancyfoot[OC]{\begin{flushright}\thepage\end{flushright}}
}

% fancy headers for the thesis
\fancyhead{}
\fancyhead[LE]{\slshape \nouppercase \leftmark}
\fancyhead[RO]{\slshape \nouppercase \rightmark}
\fancyfoot[EC]{\begin{flushleft}\thepage\end{flushleft}}
\fancyfoot[OC]{\begin{flushright}\thepage\end{flushright}}
\renewcommand{\headrulewidth}{0.4pt}
\setlength{\headheight}{14pt}

\@ifundefined{showcaptionsetup}{}{%
 \PassOptionsToPackage{caption=false}{subfig}}
\usepackage{subfig}
\makeatother

\usepackage{listings}
\renewcommand{\lstlistingname}{Listing}

\begin{document}
\frontmatter
\pagestyle{empty}
\newgeometry{margin=3cm}\begin{titlepage}

\begin{center}
\Large\textbf{{\textsc{POLITECNICO DI MILANO}}}\\
\Large{Scuola di Ingegneria Industriale e dell'Informazione}\\
\large{Corso di Laurea Magistrale in Ingegneria Spaziale}\\
\large{Dipartimento di Scienze e Tecnologie Aerospaziali (DAER)}
\par\end{center}

\vspace{0.5cm}

\begin{center}
\begin{figure}[h]
\centering{}\includegraphics[width=0.3\textwidth]{title-page/logo-polimi}
\end{figure}
\vspace{0.5cm}
\par\end{center}

\begin{center}
\textbf{\LARGE{Analysis of a Vision-Based pose initialization algorithm for non-cooperative spacecraft on synthetic imagery}}\vspace{0.5cm}
\vspace{0.2cm}
\par\end{center}

\begin{center}
\textbf{D-Orbit}\\
\textit{in collaboration with}\\
\textbf{Space Missions Engineering Lab}
\end{center}\vspace{1.5cm}

\begin{flushleft}
\begin{tabular}{ll}
Relatore:  & Prof. Jhon DOE\tabularnewline
Correlatore: & Dott. Ing. Jhon DOE\tabularnewline
\end{tabular}\vspace{1.8cm}
\par\end{flushleft}

\begin{flushright}
\begin{tabular}{ll}
Tesi di laurea di: & \tabularnewline
Francescodario CUZZOCREA & Matr. 885016\tabularnewline
\end{tabular}\vspace{1.5cm}
\par\end{flushright}

\begin{center}
{\large{}Anno Accademico 2019-2020}{\large\par}
\par\end{center}

\end{titlepage}
\restoregeometry

\cleardoublepage{}

\begin{flushright}
\emph{To someone very special\ldots{}
}\cleardoublepage{}
\par\end{flushright}

\chapter*{Acknowledgments}

\thispagestyle{empty}%«È di cattivo gusto ringraziare il relatore. Se vi ha aiutato ha fatto solo il suo dovere» Umberto Eco, Come si fa una tesi di laurea
Ebbene si, alla fine cel'ho fatta anche io ad arrivare alla fine di questo lungo percorso universitario, iniziato in un soleggiato Settembre del 2009. Di questo devo molto a mio padre e mio fratello, che mi hanno supportato durante tutto questo tempo. Devo però ammettere che non sempre ho creduto di farcela, e cretedemi, non è solo una frase di circostanza, lo ho davvero creduto, sopratutto nei momenti più bui. Arrivare a questo punto non è stato per niente facile e mi è costato tanto, non solo in termini di salute fisica e mentale, ma anche e sopratutto in termini di rapporti umani. Diversi sono i rapporti che ho perso a causa del mio essere costantemente preoccupato e spaventato per qualche esame, o, ultimamente, per la tesi. Fortunatamente però, sono anche riuscito a circondarmi di tante altre persone che, chi più chi meno, mi hanno dato qualcosa e mi sono state vicine. In particolare, un pensiero speciale e la mia immensa gratitudine vanno a Ilaria Cannizzaro e Jacopo Guarneri per il supporto tecnico e sopratutto morale che mi hanno fornito durante questa avventura trascorsa insieme in D-Orbit, specie mentre si discuteva di \textit{"rotazioni"}. Come non ringraziare inoltre il mitologico Flavio, che mi è sempre stato vicino, sopratutto nei momenti più difficili della vita, e come non ricordare Aureliano, Federico, Mattia, Umberto e Trevis che mi hanno supportato sin dal lontano 2012. Ma anche Viviana e Andrea, sempre presenti con me nella Biblioteca di Aerospaziale a studiare! Per non parlare dele famose "pause accademiche" di Stefano e Davide durante le lunghe sessioni di studio pre-esame!! E' anche doveroso da parte mia dover ringraziare Alfonso e Benedetto, con i quali ho condiviso tutto il percorso della Laurea Magistrale. A tutte queste persone conosciute in Università voglio dedicare il mio grazie per essermi state vicine lungo tutto il mio percorso universitario e per avermi lasciato, chi più chi meno, qualcosa di loro. Ancora qualche riga voglio spenderla per mandare un pensiero speciale ad Amro, Anas, Antonio, Emilio, Gian Marco, Mayra e tutti i ragazzi che ho conosciuto in Biblioteca durante la stesura finale di questo lavoro. Grazie a tutti voi, ogni singola goccia di sudore spesa su questa tesi è stata accompagnata da un sorriso. Un debito ringraziamento voglio inoltre rivolgerlo a tutti i professori che in questi anni di Politecnico mi sono stati vicini, in particolare, i proff. Colombo, Mantegazza e Quartapelle meritano i miei ringraziamenti più sentiti per avermi aiutato a capire che tipo di strada intraprendere quando le mie idee erano confuse.
\newpage
Infine, ma non certo ultimi per ordine di importanza, voglio ringraziare Cecilia e la sua famiglia per essermi stati accanto negli anni passati.

\vspace{\baselineskip}

\textit{Francescodario Cuzzocrea}, \today, Milano\cleardoublepage{}

\chapter*{Abstract}

\thispagestyle{empty}%The abstract must contain 3 main logic blocks which will be discussed in the introduction.

%Field of work
%The first block must contain a sentence which describes the field of work, and eventually another sentence which focuses the specific objective of the work in details.

%Purpose of the thesis
%The second block must start with the words «The purpose of the thesis is …».

%Short recap
%The last block must summarize the conducted activities and the obtained results (evaluating them eventually).

Nowadays...\cleardoublepage{}

\chapter*{Sommario}

\thispagestyle{empty}%Il testo delle tesi redatte in lingua straniera dovrà essere introdotto da un ampio estratto in lingua italiana, che andrà collocato dopo l’abstract. 
\cleardoublepage\pagenumbering{roman}
\setcounter{page}{1}
\pagestyle{fancy}\tableofcontents{}\listoffigures
\listoftables
\listof{algorithm}{List of Algorithms}
\cleardoublepage\mainmatter


% Symbols section
\clearpage
\printnoidxglossary[title=List of Symbols]

% Acronym section
\clearpage
\printnoidxglossary[type=\acronymtype,title=Abbreviated Terms]

%************************************************************
% Defining useful commands
%************************************************************
\renewcommand{\sectionmark}[1]{\markright{\thesection.\ #1}}
\renewcommand{\chaptermark}[1]{\markboth{\thechapter.\ #1}{}}
\newcommand{\norm}[1]{\left\lVert#1\right\rVert}

\chapter{Introduction\label{chap:first-chapter}}

\begin{quotation}
{\footnotesize
\noindent{\emph{``We are just an advanced breed of monkeys on a minor planet of a very average star. But we can understand the Universe. That makes us something very special.''\\}
}
\begin{flushright}
Stephen Hawking
\end{flushright}
}
\end{quotation}
\vspace{0.5cm}

\section{State of Art}

\subsection{Synthetic image generation}

\subsection{Pose Estimation}

\subsubsection{Pose estimation sensors}

\subsubsection{Pose estimation tecniques}

%\cite{Dijkstra68Letters}.


\chapter{State-of-the-art and Limitations\label{chap:second-chapter}}

\begin{quotation}
{\footnotesize
\noindent{\emph{``All models are wrong, but some are useful.''\\}
}
\begin{flushright}
George Box
\end{flushright}
}
\end{quotation}
\vspace{0.5cm}


\chapter{Synthetic image generation : \label{chap:third-chapter}}

\begin{quotation}
{\footnotesize
\noindent{\emph{``In math, you're either right or you're wrong.''\\}
}
\begin{flushright}
Katherine Johnson
\end{flushright}
}
\end{quotation}
\vspace{0.5cm}

\section{POV-Ray}

\subsection{Raytracing}

\subsection{Program's features}

\subsection{Light source ?}

\subsection{Camera setup ?}

\subsubsection{camera attitude matrix ?}

\subsection{Earth setup ?}

\subsection{Spacecraft Modeling ?}
\subsubsection{Blender ?}

\section{Matlab integration ?}



\chapter{The SVD architecture for pose initialization\label{chap:fourth-chapter}}

\begin{quotation}
{\footnotesize
\noindent{\emph{``Prediction is very difficult, especially if it's about the future.''\\}
}
\begin{flushright}
Niels Bohr
\end{flushright}
}
\end{quotation}
\vspace{0.5cm}

\section{Mathematical Preliminaries}

\subsection{Pinhole Camera Model}
%review of solution methods

\subsection{Image derivatives}
%review of solution methods

\subsection{The Hough Transform}
%review of solution methods

\subsection{Perspective-n-Point problem}
%review of solution methods

\section{Feature-based pose estimation implementation}

\subsection{Architecture}

\subsection{Image processing subsystem}

\subsubsection{Feature Detection}

\subsubsection{Feature Synthesis}

\subsection{Pose Determination}

\subsubsection{Feature Correspondence}

\subsubsection{Pose Refinement}




\chapter{Results\label{chap:sixth-chapter}}

\begin{quotation}
{\footnotesize
\noindent{\emph{``Most good programmers do programming not because they expect to get paid or get adulation by the public, but because it is fun to program.''\\}
}
\begin{flushright}
Linus Torvalds
\end{flushright}
}
\end{quotation}
\vspace{0.5cm}

\section{Syntetic Dataset Validation}


\chapter*{Conclusions\label{chap:conclusion}}

\addcontentsline{toc}{chapter}{Conclusions}
\markboth{Conclusions}{Conclusions}%The conclusions must recall the field of work, the purpose of the thesis, what has been done and an evaluation of the obtained results.
%Furthermore, the conclusion must also emphasize the future prospects and must show how to move forward in the study area.
\begin{quotation}
  {\footnotesize
    \noindent{\emph{``If you didn't get angry and mad and frustrated, that means you don't care about the end result, and are doing something wrong.''\\}
    }
    \begin{flushright}
      Greg Kroah-Hartman
    \end{flushright}
  }
\end{quotation}
\vspace{0.5cm}

This project was successful in developing a tool which allows the end user to produce hundreds of images of a target \acrshort{sc} given its \acrshort{stl} model, using the open source ray tracer \acrshort{povray} in conjunction with MATLAB. The full uncontrolled dynamics of the target \acrshort{sc} has been simulated as well, so, if the inertial properties of the target are known, is possible to closely simulate the dynamical behavior of the target itself and from that, generate the images. Optionally is possible to also insert the Earth in the background, which enhances the realism of the scene. In particular, a great care has been putted into taking in consideration both the optical properties of the Earth surface and of the \acrshort{sc} surface.
The generated data-set has been then compared to the freely available SPEED data-set both in terms of a qualitative assessment (by comparing the histograms) and by applying a \acrshort{cv} algorithm.
The preliminary results obtained from the comparison shows that similar results have been obtained.
Still, there is room for improvement. Such as enhancing the model of the Earth to make it comparable with what found on the SPEED data-set. For what concerns the model of the \acrshort{sc} instead, more realistic texture could be used, if available. A better modeled surface would result in more accurate images produced and a better study could be made to better model the optical properties of the materials which covers the \acrshort{sc}. Moreover, overall rendering times when the Earth is in background are really long, as Earth rendering is a high demanding application. A possible improvement could be to find a way to cut the texture and Earth using not the full sphere but a subset with a mesh, for example. Or better, to patch the ray tracer source code to allow performing the rendering step on the GPU.
Concerning the image analysis instead, different pose solvers could be tried. In this work, a MATLAB builtin function has been used. Different pose solver could be used, such as the \textit{e}\acrshort{pnp} \cite{10.1007/s11263-008-0152-6} to make a comparison in terms of efficiency. Another improvement could be to employ separated Hough transforms to detect different geometric shapes, as suggested also in \cite{Sharma2018}. In general, the image processing subsystem still has to be improved, as it is susceptible to producing spurious edges, especially when the target is far distant from the camera or when the solar panel are not in shadow (so, when they are hitted directly from the Sun's light).
Imagining to implement the algorithm on an embedded board, the proposed toolbox also allows the set-up of hardware in the loop experiments to evaluate the computational time on real H/W.
Moreover, having available a toolbox capable of producing hundreds of images at will, also enables to implement completely different strategies for pose determination, such as the one based on neural networks like for example what proposed in \cite{Sharma2019} and compare the results with what obtained implementing more traditional techniques such as the ones used in this project.\bibliographystyle{plain}

\appendix

\chapter{First appendix: Attitude and Orbit Simulator \label{app:first-appendix}}

In this appendix the kinematic and dynamic representation used for the simulation of the attitude of the target spacefrat will briefly be descripted.

\section{Keplerian Motion}
The dynamics of the \acrshort{cg} of a rigid body is given by Newton equation of motion stating that the variation in time of the linear mentum is equal to the external forces applied to the body, with respect to an inertial reference frame. Thus :

\begin{equation}
  \frac{d}{dt} (\textit{m} \mathbf{v}) = \sum\limits_{i=1}^n \mathbf{f_i}
\end{equation}

where $\textit{m}$ is the mass of the body, $\mathbf{v}$ its velocity and $\mathbf{f_i}$ the forces applied to the system. Then, for a rigid body orbiting around a single massive body, the main force to take into account is the gravitational pull that can be modelled accordingly to Newton law of gravitation as follows :

\begin{equation}
  \label{eqn:orbiteqtn}
  m \mathbf{\ddot{r}} = - \frac{\mu \textit{m}} {\norm{\mathbf{r}}^3} \mathbf{r} + \mathbf{f}
\end{equation}

where $\mu$ is the gravitational constant of the main attractor, $\mathbf{r}$ the position vector of the orbiting body computed from the main attractor \acrshort{cg} and $\mathbf{f}$ the sum of other forces applied to the system. The underlying assumption is to consider the main attractor to be still or moving in linear constant motion, which is never the case. Such assumptions holds well enough for many problem of interest. Considering null or negligible other forces it follows that the system motion is central, thus the angular momentum is conserved.\\
Such quantity, for this system, is defined as follows

\begin{equation}
  \mathbf{h} = \mathbf{r} \wedge \mathbf{r}
\end{equation}

where $\mathbf{v}$ is the velocity, derivative of the position vector $\mathbf{r}$ expressed in the said reference frame.\\
Crossing \ref{eqn:orbiteqtn} with the angular momentum and considering that the gravitational field is conservative, it is possible to determine the position of the orbiting body in time through six parameters that represent the analytical solution of the restricted two body problem. For this research the influence of other massive bodies is not taken into account as for many of the applications here considered can be analysed considering satellites to be small bodies close to the planet.\\
The most used set of parameters are often referred to as Keplerian parameters, however different parametrization are possible. Of these six constant two represent the shape of the orbit, which must be a conic according to Kepler's studies and Newton's formulation, three identify the orientation of the orbit in a \acrshort{3d} space and the last one links the position along the orbital with time. The shape is identified by the semi-major axis a of the conic and the eccentricity e, restricted to be between zero and one for closed orbits, being zero for circular orbits. In the reference frame with z axis aligned with angular momentum and x axis aligned with the minimum orbital distance (eccentricity vector) the position vector can be written as follows

\begin{equation}
  \mathbf{r_{orb}} = \frac{\nicefrac{\norm{h}^2}{\mu}}{1 + e cos \theta} \begin{Bmatrix} cos \theta
    \\ sin \theta
    \\ 0
  \end{Bmatrix} = \frac{a(1-e^2)}{1+ e cos \theta} \begin{Bmatrix} cos \theta
    \\ sin \theta
    \\ 0
  \end{Bmatrix}
\end{equation}

Then the position vector $\mathbf{r_{orb}}$ in this frame can be translated back into the inertial $\mathbf{r}$ by using three sequential rotations in the order \textit{z} - \textit{x} - \textit{z} according to the values of the other three parameters: argument of pericenter $\omega$ , inclination \textit{i} and right ascension of the ascending node $\Omega$.


\section{Dynamics}

\subsection{Euler Equations}

The dynamics of a rigid body which is tumbling into space can be modeled by mean of the well known Euler equations, having in mind that such equation is expressed in a non-inertial reference frame attached to the body frame :

\begin{equation}
  \dot{\mathbf{\omega_T}} = \mathbf{(I_T)}^{-1} \left[\mathbf{L_T} - \mathbf{\omega_T}  \wedge (\mathbf{I_T} \mathbf{\omega_T})\right]
\end{equation}

where \textbf{$\omega_T$} is the angular velocity vector of the spacecraft in body frame, \textbf{$I_T$} is the inertia tensor of the spacecraft in body frame and \textbf{$L_T$} are the external torques acting on the spacecraft due to  external disturbances and control torques.

\section{Direct Cosine Matrix}
The direct cosine matrix, or attitude matrix, gives the transformation of a vector from a reference frame $N$ to another reference frame $O$ :

\begin{equation}
  \mathbf{r_{O}} = \mathbf{A_{ON}} \mathbf{r_{N}}
\end{equation}

So, if we consider the spacecraft body frame and the inertial frame, then we can write the relation between the two frames as :

\begin{equation}
  \mathbf{r_{T}} = \mathbf{A_{TN}} \mathbf{r_{N}}
\end{equation}

As it is demonstrated in Ref. \cite{Markley2014}, we can express the time dependence of the attitude matrix by writing the rotation from the body frame to the inertial frame as :

\begin{equation}
  \dot{A}_{TN}= - [\omega_{TN} \times]A_{TN}
\end{equation}

where $[\omega_{TN} \times]$ is the skew-symmetric cross product matrix containing the components of the angular velocity vector :

\begin{equation*}
  \mathbf{[\omega \times]} =
  \begin{bmatrix}
    0                & -\omega_{TN_{3}} & \omega_{TN_{2}}  \\
    \omega_{TN_{3}}  & 0                & -\omega_{TN_{1}} \\
    -\omega_{TN_{2}} & \omega_{TN_{1}}  & 0
  \end{bmatrix}
\end{equation*}

Thus, by integrating this relation, we can get full attitude at every iteration.

\subsection{Environmental Disturbances} \label{sec:disturbances}
The external disturbances acting on spacecraft placed in a LEO orbit are basically four and are due to :

\begin{itemize}
  \item Gravity Gradient
  \item Magnetic Field
  \item Solar Radiation Pressure
  \item Aerodynamic Drag
\end{itemize}

Details about how those torques have been mathematically modeled will be given in the following sections.

\subsubsection{Gravity Gradient}
Any non-symmetrical rigid body in a gravity field is subject to a gravity-gradient torque.
If we consider a rigid spacecraft, the torque due to the gravity gradient about the spacecraft's center of mass can be modeled as :

\begin{equation}
  \mathbf{L_{gg}} = \frac{3 \ mu}{r^3} \mathbf{n} \wedge (\mathbf{I} \ \mathbf{n})
\end{equation}

where $\mu$ is the Eart's gravitational constant, $\textbf{r}$ is the radius vector from the center of the Earth, thus $r \equiv \Vert{\textbf{r}}\Vert$, $\textbf{I}$ is the inertia tensor in body frame and $\textbf{n}$ is the body frame representation of a nadir-pointing unit vector.\\
Further details about the model can be found in reference \cite{Markley2014}.

\subsubsection{Earth's Magnetic Field}
The torque generated by a magnetic dipole $\textbf{m}$ in a magnetic field $\textbf{B}$ is

\begin{equation}
  \mathbf{L_{mag}} = \mathbf{m} \wedge \mathbf{B}
  \label{eq:magfield}
\end{equation}

where $\mathbf{B}$ is the magnetic field in the body frame.
The most basic source of a magnetic dipole is a current loop. A current of I amperes flowing in a planar loop of area A produces a dipole moment of magnitude m=IA in the direction normal to the plane of the loop and satisfying a right-hand rule.
When $\textbf{m}$ is in $Am^2$ and the magnetic field is specified in Tesla, Eq. \ref{eq:magfield} gives the torque in $Nm$. If there are N turns of wire in the loop, the dipole moment has magnitude m=NIA (such as the case of a magnetic torquer).
To model $\textbf{B}$ either the full IGRF model or the simple dipole model can be used.\\
For what concerns this work, the full IGRF model truncated to the 10th order has been used. Further details about the model can be found in reference \cite{Davis2014}

\subsubsection{Solar Radiation Pressure}
Solar radiation pressure acting on the surfaces of the spacecraft causes a disturbance torque that in general, cannot be neglected for orbits higher than \SI{800}{\kilo\meter}, so it has been taken into account in this work.
The SRP torque is zero zero when the spacecraft is in the shadow of the Earth or any other body, of course.
To take into account the effect of solar radiation pressure on the spacecraft, the spacecraft's surface can be modeled as a collection of $N$ flat plates of area $S_{i}$, outward normal $\mathbf{n_{b}}$ in the body coordinate frame, specular reflection coefficient $\rho_s$, diffuse reflection coefficient $\rho_{d}$ and absorption coefficient $\rho_{a}$; those coefficients must sum to unity.
For what concerns this work, only $\rho_s$ and $\rho_d$ have been considered, since all the absorbed radiation is emitted as thermal radiation,  although not necessarily at the same time or from the same surface as its absorption.
For an accurate modeling of thermal radiation we must also known the absolute temerature and the emissivity of each surface.
We can define the spacecraft-to-Sun unit vector in the spacecraft's body frame as :

\begin{equation}
  \mathbf{s_t} = \mathbf{A_{TN}} \mathbf{s_i}
\end{equation}

where $\mathbf{A_{TN}}$ is the attitude matrix and $\mathbf{s_i}$ is the spacecraft-to-Sun unit vector in the GCI frame.
We can define the angle between the Sun vector and the normal exiting from the normal to the i-th plate as :

\begin{equation}
  cos(\theta_{SRP}^{i}) = \mathbf{n}_{T}^{i} \cdot \mathbf{s_b}
\end{equation}

The SRP force on the i-th plate can be expressed as :

\begin{equation}
  \mathbf{F}_{SRP} = - P_{Sun}A_{i}\left[ 2\left( \frac{\rho_{d}^{i}}{3} + \rho_{s}^{i}cos(\theta_{SRP}^{i}) \right) \mathbf{n}_{B}^{i} + (1 -\rho_{s}) \mathbf{s_t} \right] max(cos(\theta_{SRP}^{i}),0)
\end{equation}

where $P_{Sun}$ is the solar radiation pressure.
The Solar radiation pressure torque acting on the spacecraft is then given by :

\begin{equation}
  \mathbf{L}_{SRP}^{i} = \sum\limits_{i=1}^n  \mathbf{r}_{i} \times \mathbf{F}_{SRP}^{i}
\end{equation}

where $\mathbf{r}_{i}$ is the vector from the spacecraft center of mass to the centre of pressure of the SRP on the i-th face.
In this formulation we are not considering the albedo radiation coming from the Earth and from the Moon.
Further details on how the solar radiation pressure, the spacecraft-to-Sun unit vector and the eclipse condition have been modeled can be found in reference \cite{Markley2014}.

\subsubsection{Aerodynamic Drag}
Aerodynamic torque is generally computed by modeling the spacecraft as a collection of $N$ flat plates of area $A_i$ and outward normal unit vector $\mathbf{n_{B}}$ expressed in the spacecraft body-fixed coordinate system. The torque depends on the velocity of the spacecraft relative to the atmosphere, which is not simply the velocity of the spacecraft in the GCI frame, because the atmosphere is not stationary in that frame.
The most common assumption is that the atmosphere co-rotates with the Earth. The relative velocity in the GCI frame is then given by :

\begin{equation}
  \mathbf{v_{relI}} =  \mathbf{v_I} + [\mathbf{\omega_{\earth} \times}] \mathbf{r_I}
\end{equation}

where $\mathbf{r_I}$ and $\mathbf{v_I}$ are the position and velocity of the spacecraft expressed in the GCI frame.
The Earth's angular velocity vector is $\mathbf{\omega_{\earth}} = \omega_{\earth}[0 0 1]'$ with $\omega_{\earth}$ = \SI{0.000072921158553}{\radian/\second}.
The velocity in the body frame is the computed as :

\begin{equation}
  \mathbf{v_{relT}} = \mathbf{A_{TN}}   \begin{bmatrix} \dot{x} + \omega_{\earth} y \\ \dot{y} - \omega_{\earth} x \\ \dot{x} \end{bmatrix}
\end{equation}

The inclination of the \textit{i-th} plate with respect to the relative velocity is given by :

\begin{equation}
  cos(\theta_{aero}^{i}) = \frac{\mathbf{n}_{T}^{i} \cdot \mathbf{v}_{rel\ T}}{\norm{\mathbf{v}_{rel}}}
\end{equation}

The aerodynamic force on i-th plate in the flat plate model is

\begin{equation}
  \mathbf{F}_{aero}^{i} = - \frac{1}{2} \rho C_{D} \norm{\mathbf{v}_{rel}} \mathbf{v}_{rel_{} T}\ S_{i}\ max(cos(\theta_{aero}^{i}),0)
\end{equation}

where $\rho$ is the atmospheric density, and $C_D$ is the drag coefficient.
$\rho$ can be modeled by mean of the well known exponential decaying model for the atmospheric density :

\begin{equation}
  \rho = \rho_{0} e^{(-(h-h_{0})/H)}
\end{equation}

where $\rho_{0}$ and $h_{0}$ are reference density and height, respectively, $h$ is the height above the ellipsoid and $H$ is the scale height, which is the fractional change in density with eight.
The value of $\rho_{0}$, $h_{0}$  and $H$ changes with $h$.
The values used to perform the simulation are the one given in \cite{Markley2014}.
The actual torque due to the aerodynamic drag can be computed as :

\begin{equation}
  \mathbf{L}_{aero}^{i} = \sum\limits_{i=1}^n  \mathbf{r}_{i} \times \mathbf{F}_{aero}^{i}
\end{equation}

where $n$ is the number of faces and $\mathbf{r}_{i}$ is the vector from the spacecraft center of mass to the center of pressure on the $i$ th face.

\chapter{Second appendix: Random Rotation Matrix Generation \label{app:second-appendix}}

\section{Kinematics}
In this section the kinematics representation used for the simulation will briefly be descripted. Two approach has been taken : 
\begin{itemize}
 \item [-] The DCM representation
 \item [-] The quaternion representation
\end{itemize}

\subsection{Direct Cosine Matrix}
The direct cosine matrix, or attitude matrix, gives the transformation of a vector from a reference frame $N$ to another reference frame $O$ : 

\begin{equation}
 \mathbf{r_{O}} = \mathbf{A_{ON}} \mathbf{r_{N}}
\end{equation}

So, if we consider the spacecraft body frame and the inertial frame, then we can write the relation between the two frames as : 

\begin{equation}
 \mathbf{r_{B}} = \mathbf{A_{B/N}} \mathbf{r_{N}}
\end{equation}

As it is demonstrated in Ref. \cite{Markley2014}, we can express the time dependence of the attitude matrix by writing the rotation from the body frame to the inertial frame as : 

\begin{equation}
 \dot{A}_{B/N}= - [\omega_{B/N} \times]A_{B/N} 
\end{equation}

where $[\omega_{B/N} \times]$ is the skew-symmetric cross product matrix containing the components of the angular velocity vector : 

\begin{equation*}
 \mathbf{[\omega \times]} =
                                \begin{bmatrix}
                                    0 & -\omega_{B/N_{3}} & \omega_{B/N_{2}} \\
                                    \omega_{B/N_{3}} & 0 & -\omega_{B/N_{1}} \\
                                    -\omega_{B/N_{2}} & \omega_{B/N_{1}} & 0
                                \end{bmatrix}
\end{equation*}

Thus, by integrating this relation, we can get full attitude at every iteration.

\subsection{Quaternions}
Quaternions are another way to parametrize the attitude of a rigid body in space. They are commonly used for control purpose due to the fact that they don't have kinematics singularities.
A quaternion is a four-component vector with some additional operations defined on it. A quaternion $\mathbf{q}$ has a three-vector part $\mathbf{q_{1:3}}$ and a scalar part $q_{4}$ : 

\begin{equation*}
 \mathbf{q} =
                                \begin{bmatrix}
                                    q_{1}\\
                                    q_{2}\\
                                    q_{3}\\
                                    q_{4}\\
                                \end{bmatrix}
\end{equation*}

The kinematic equation for the quaternion can be written as :

\begin{equation}
 \label{eq:quaternion}
 \dot{q} = \frac{1}{2} \mathbf{\varXi(\mathbf{q})} \mathbf{\omega_{B/N}}
\end{equation}

where $\mathbf{\varXi(\mathbf{q})}$ is defined as 

\begin{equation*}
 \mathbf{\varXi(\mathbf{q})} =
                                \begin{bmatrix}
                                   q_4 & -q_3 & q_2\\
                                   q_3 & q_4 & -q_1\\
                                  -q_2 & q_1 & q_4\\
                                  -q_1 & -q_2 & -q_3\\
                                \end{bmatrix}
\end{equation*}

By integrating Eq. \ref{eq:quaternion} can get the full attitude at every iteration in terms of quaternion.

\section{Dynamics}
\subsection{Euler Equations}
The dynamics of a rigid body which is tumbling into space can be modeled by mean of the well known Euler equations : 

\begin{equation}
  \mathbf{\omega_B} = \mathbf{(I_B)}^{-1} [\mathbf{L_B} - \mathbf{\omega_b}  \wedge (\mathbf{I_B} \mathbf{\omega_b})
\end{equation}

where \textbf{$\omega_B$} is the angular velocity vector of the spacecraft in body frame, \textbf{$I_B$} is the inertia tensor of the spacecraft in body frame and \textbf{$L_B$} are the external torques acting on the spacecraft due to  external disturbances and control torques.

\subsection{Environmental Disturbances} \label{sec:disturbances}
The external disturbances acting on spacecraft placed in a LEO orbit are basically four and are due to :

\begin{itemize}
  \item[-] Gravity Gradient
  \item[-] Magnetic Field 
  \item[-] Solar Radiation Pressure
  \item[-] Aerodynamic Drag
\end{itemize}

Details about how those torques have been mathematically modeled will be given in the following sections.

\subsubsection{Gravity Gradient}
Any non-symmetrical rigid body in a gravity field is subject to a gravity-gradient torque.
If we consider a rigid spacecraft, the torque due to the gravity gradient about the spacecraft's center of mass can be modeled as :

\begin{equation}
 \mathbf{L_{gg}} = \frac{3 \ mu}{r^3} \mathbf{n} \wedge (\mathbf{I} \ \mathbf{n})
\end{equation}

where $\mu$ is the Eart's gravitational constant, $\textbf{r}$ is the radius vector from the center of the Earth, thus $r \equiv \Vert{\textbf{r}}\Vert$, $\textbf{I}$ is the inertia tensor in body frame and $\textbf{n}$ is the body frame representation of a nadir-pointing unit vector.\\
Further details about the model can be found in reference \cite{Markley2014}.

\subsubsection{Earth's Magnetic Field}
The torque generated by a magnetic dipole $\textbf{m}$ in a magnetic field $\textbf{B}$ is

\begin{equation}
 \mathbf{L_{mag}} = \mathbf{m} \wedge \mathbf{B}
 \label{eq:magfield}
\end{equation}

where $\mathbf{B}$ is the magnetic field in the body frame.
The most basic source of a magnetic dipole is a current loop. A current of I amperes flowing in a planar loop of area A produces a dipole moment of magnitude m=IA in the direction normal to the plane of the loop and satisfying a right-hand rule.
When $\textbf{m}$ is in $Am^2$ and the magnetic field is specified in Tesla, Eq. \ref{eq:magfield} gives the torque in $Nm$. If there are N turns of wire in the loop, the dipole moment has magnitude m=NIA (such as the case of a magnetic torquer).
To model $\textbf{B}$ either the full IGRF model or the simple dipole model can be used.\\
For what concerns this work, the full IGRF model truncated to the 10th order has been used. Further details about the model can be found in reference \cite{Davis2014}

\subsubsection{Solar Radiation Pressure}
Solar radiation pressure acting on the surfaces of the spacecraft causes a disturbance torque that in general, cannot be neglected for orbits higher than \SI{800}{\kilo\meter}, so it has been taken into account in this work.
The SRP torque is zero zero when the spacecraft is in the shadow of the Earth or any other body, of course.
To take into account the effect of solar radiation pressure on the spacecraft, the spacecraft's surface can be modeled as a collection of $N$ flat plates of area $S_{i}$, outward normal $\mathbf{n_{b}}$ in the body coordinate frame, specular reflection coefficient $\rho_s$, diffuse reflection coefficient $\rho_{d}$ and absorption coefficient $\rho_{a}$; those coefficients must sum to unity.
For what concerns this work, only $\rho_s$ and $\rho_d$ have been considered, since all the absorbed radiation is emitted as thermal radiation,  although not necessarily at the same time or from the same surface as its absorption.
For an accurate modeling of thermal radiation we must also known the absolute temerature and the emissivity of each surface.
We can define the spacecraft-to-Sun unit vector in the spacecraft's body frame as : 

\begin{equation}
 \mathbf{s_b} = \mathbf{A_{B/N}} \mathbf{s_i}
\end{equation}

where $\mathbf{A_{B/N}}$ is the attitude matrix and $\mathbf{s_i}$ is the spacecraft-to-Sun unit vector in the GCI frame.
We can define the angle between the Sun vector and the normal exiting from the normal to the i-th plate as : 

\begin{equation}
 cos(\theta_{SRP}^{i}) = \mathbf{n}_{B}^{i} \cdot \mathbf{s_b}
\end{equation}

The SRP force on the i-th plate can be expressed as : 

\begin{equation}
 \mathbf{F}_{SRP} = - P_{Sun}A_{i}\left[ 2\left( \frac{\rho_{d}^{i}}{3} + \rho_{s}^{i}cos(\theta_{SRP}^{i}) \right) \mathbf{n}_{B}^{i} + (1 -\rho_{s}) \mathbf{s_b} \right] max(cos(\theta_{SRP}^{i}),0)
\end{equation}

where $P_{Sun}$ is the solar radiation pressure.
The Solar radiation pressure torque acting on the spacecraft is then given by :

\begin{equation}
    \mathbf{L}_{SRP}^{i} = \sum\limits_{i=1}^n  \mathbf{r}_{i} \times \mathbf{F}_{SRP}^{i} 
\end{equation}

where $\mathbf{r}_{i}$ is the vector from the spacecraft center of mass to the centre of pressure of the SRP on the i-th face.
In this formulation we are not considering the albedo radiation coming from the Earth and from the Moon.
Further details on how the solar radiation pressure, the spacecraft-to-Sun unit vector and the eclipse condition have been modeled can be found in reference \cite{Markley2014}.

\subsubsection{Aerodynamic Drag}
Aerodynamic torque is generally computed by modeling the spacecraft as a collection of $N$ flat plates of area $A_i$ and outward normal unit vector $\mathbf{n_{B}}$ expressed in the spacecraft body-fixed coordinate system. The torque depends on the velocity of the spacecraft relative to the atmosphere, which is not simply the velocity of the spacecraft in the GCI frame, because the atmosphere is not stationary in that frame.
The most common assumption is that the atmosphere co-rotates with the Earth. The relative velocity in the GCI frame is then given by : 

\begin{equation}
 \mathbf{v_{relI}} =  \mathbf{v_I} + [\mathbf{\omega_{\earth} \times}] \mathbf{r_I}
\end{equation}

where $\mathbf{r_I}$ and $\mathbf{v_I}$ are the position and velocity of the spacecraft expressed in the GCI frame. 
The Earth's angular velocity vector is $\mathbf{\omega_{\earth}} = \omega_{\earth}[0 0 1]'$ with $\omega_{\earth}$ = \SI{0.000072921158553}{\radian/\second}.
The velocity in the body frame is the computed as : 

\begin{equation}
 \mathbf{v_{relB}} = \mathbf{A_{B/N}}   \begin{bmatrix} \dot{x} + \omega_{\earth} y \\ \dot{y} - \omega_{\earth} x \\ \dot{x} \end{bmatrix}
\end{equation}

The inclination of the i-th plate WRT the relative velocity is given by : 

\begin{equation}
 cos(\theta_{aero}^{i}) = \frac{\mathbf{n}_{B}^{i} \cdot \mathbf{v}_{rel\ B}}{\norm{\mathbf{v}_{rel}}}
\end{equation}

The aerodynamic force on i-th plate in the flat plate model is

\begin{equation}
 \mathbf{F}_{aero}^{i} = - \frac{1}{2} \rho C_{D} \norm{\mathbf{v}_{rel}} \mathbf{v}_{rel_{} B}\ S_{i}\ max(cos(\theta_{aero}^{i}),0)
\end{equation}

where $\rho$ is the atmospheric density, and $C_D$ is the drag coefficient.  
$\rho$ can be modeled by mean of the well known exponential decaying model for the atmospheric density :

\begin{equation}
 \rho = \rho_{0} e^{(-(h-h_{0})/H)}
\end{equation}

where $\rho_{0}$ and $h_{0}$ are reference density and height, respectively, $h$ is the height above the ellipsoid and $H$ is the scale height, which is the fractional change in density with eight.
The value of $\rho_{0}$, $h_{0}$  and $H$ changes with $h$. 
The values used to perform the simulation are the one given in \cite{Markley2014}.
The actual torque due to the aerodynamic drag can be computed as : 

\begin{equation}
 \mathbf{L}_{aero}^{i} = \sum\limits_{i=1}^n  \mathbf{r}_{i} \times \mathbf{F}_{aero}^{i}
\end{equation}

where $n$ is the number of faces and $\mathbf{r}_{i}$ is the vector from the spacecraft center of mass to the center of pressure on the $i$ th face.
\cleardoublepage

%\chapter{Third appendix\label{app:third-appendix}}

%\input{appendix-c/appendix-c.tex}\cleardoublepage

\bibliography{bibliography}
\addcontentsline{toc}{chapter}{Bibliography}

\end{document}
