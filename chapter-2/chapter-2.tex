\begin{quotation}
{\footnotesize
\noindent{\emph{``We are just an advanced breed of monkeys on a minor planet of a very average star. But we can understand the Universe. That makes us something very special.''\\}
}
\begin{flushright}
Stephen Hawking
\end{flushright}
}
\end{quotation}
\vspace{0.5cm}


\section{Synthetic Image Generation for Spaceborne Applications}

\subsection{Professional Solutions}

\subsubsection{ESA PANGU}

PANGU is a software developed in order to create synthetic planetary surface images, as much representative as possible, to aid the development of vision-based algorithms.\\
PANGU is a nice software which is ready to use. It's use is permitted for free for users working on an ESA project, while non-ESA users have to contact STAR-Dundee to purchase PANGU with technical support.\\
PANGU can also be integrated with proprietary or OSS simulation tools and it gives the possibility to correctly simulate a space camera in all its aspects (so focal lengths, and other relevant parameters of the image). It renders the images using OpenGL and it can use GPU cores to accelerate the rendering.

\subsubsection{Airbus Surrender}
SurRender is a software developed by Airbus Defense and Space. The software handles various space objects such as planets, asteroids, satellites and spacecraft.\\
It is capable of accommodating solar system-sized scenes without precision loss, and optimizes the ray tracing process to explicitly target objects. It can operate in real time mode to be coupled with  proprietary or OSS simulation tools and it gives the possibility to have an Hardware in The Loop simulation to test the responsiveness of the image processing subsystem. It gives the possibility to correctly simulate a space camera in all its aspects (so focal lengths, and other relevant parameters of the image). It parallelized and so can be ran on cloud platform to accelerate rendering times.\\

\subsection{Low-Cost Solutions}

\subsubsection{SPEED dataset: image generation using OpenGL}

\subsubsection{URSO dataset: image generation using Unreal Engine 4}

\section{Spaceborne Close-Proximity Relative Navigation}

\subsection{Pose estimation sensors}

\subsection{Pose estimation tecniques}


